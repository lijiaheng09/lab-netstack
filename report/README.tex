\documentclass[a4paper]{article}

\usepackage{geometry}
\usepackage{indentfirst}

\geometry{scale=0.8}

\title{PKU-CompNet (H) Fall'22 \\ Lab Assignment (Premium): Protocol Stack \\ Report}
\author{Jiaheng Li (2000013054)}

\begin{document}
  \maketitle

  \section{Lab 1: Link-layer}

  \subsection{Writing Task 1 (WT1)}

  The third frame in the results is:
  \begin{table}[!ht]
    \centering
    \fontsize{8pt}{8pt}
    \ttfamily
    \begin{tabular}{lllllll}
      No. & Time & Source & Destination & Protocol & Length & Info \\
      12 & 1.068164 & 0.0.0.0 & 255.255.255.255 & DHCP & 342 & DHCP Discover - Transaction ID 0x13699715
    \end{tabular}
  \end{table}

  \begin{enumerate}
    \item There are 827 frames in the filtered result. (\texttt{Dislayed: 827})
    \item The destination address of the Ethernet frame is \texttt{ff:ff:ff:ff:ff:ff}, which is the broadcast address.
    \item The 71st byte is \texttt{0x15}.
  \end{enumerate}

  \subsection{Programming Task 1 (PT1)}

  The required functions are implemented in \texttt{src/lib/device.c}, which maintains a linear structure of \texttt{Device} objects declared in \texttt{src/include/internal/device.h}, containing basic information of the devices and their runtime handlers for management.

  \subsection{Programming Task 2 (PT2)}

  The required functions are implemented in \texttt{src/lib/packetio.c}.

  For a \texttt{sendFrame()} call, it simply get information from the \texttt{devices} structure, allocate space to put each element of an Ethernet II frame in order, and then call \texttt{pcap\_sendpacket()} to send it out.

  For \texttt{setFrameReceiveCallback()}, it simply modifies the global pointer to the callback.

  To receive frames, a thread is created in each successful \texttt{addDevice()}, continously running \texttt{pcap\_loop()} to call the receiving callback for that device.

  The global data about devices and the callback are protected through atomic variables or mutexes.

  \subsection{Checkpoint 1 (CP1)}
  
  The typescript record is at \texttt{checkpoints/CP1/\{typescript, timing.log\}}.

  For convenient testing, a \texttt{console} tool is implemented in \texttt{src/tools/console.c} and will be built to \texttt{build/tools/console}, which enables us to call the functions above interactively.

  In the script, several calls to \texttt{addDevice()} and \texttt{findDevice()} in a row shows that they can correctly add (only) the valid Ethernet devices on the host and correctly find the added devices.

  \subsection{Checkpoint 2 (CP2)}

  The typescript record is at \texttt{checkpoints/CP2/\{typescript, timing.log\}}.

  A testing tools is implemented in \texttt{src/tools/eth\_test.c}.
  It receives several devices and their destination address (the MAC address on their other sides) as arguments.
  It sends $10000$ random raw Ethernet frames through each device (in a new thread for each device), and receives the frames from the other side at the same time.
  (To prevent losses, it will delay for a short time after each batch of frames.)
  In the end, it will print a checksum of frames it sends, and another of frames it receives.

  In the script, the \texttt{eth\_test} tool above is simultaneously run on each of the 5 hosts in the example network provided in \texttt{vnetUtils}, using all of the 8 devices and 4 links among them.
  After the run, we can see that the sending and receiving checksum of each link matches, which shows the (basic) correctness of our sending and receiving implementation.

\end{document}
